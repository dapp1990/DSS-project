\subsection{Iteration 1 \& 2: Image Filtering}
We concluded that the best way to add this new functionalities was creating a new instantiation of  \textit{FilterAdapter}.
\par
Because we were constrained by the external configuration file, \textit{FilterReplaceImage} operates as the coordinator between the  \textit{ConfigurationReader} and the concrete implementations of \textit{SimpleImageFilter}.
\par
\textit{ConfigurationReader} is implemented as a factory class. It parses the configuration file and creates instance of \textit{SimpleImageFilter} accordingly.
\par
One of the important design decisions was that the configuration file contains the name of the concrete classes of  \textit{SimpleImageFilter} and  the order in which the filter are applied is established by the same order of the configuration file.
\par
For iteration 1 and 2, we added a zap filter by extending \textit{FilterAdaptor}. The subclass is \textit{FilterReplaceImage} and overrides the textit{onHttpResponseReceive} method. \textit{FilterAdaptor}. \textit{FilterReplaceImage} extracts the \textit{BufferedImage} from the \textit{HttpMessage}. It finds all the filter operations to apply and passes the \textit{BufferedImage} on to each of them in turn to be modified.
The actual filters are subclasses of \textit{SimpleImageFilter}. The \textit{ConfigurationReader} finds them, given the url of the config file.

\subsection{Iteration 3: Content Filtering}
We separated the distinct responsibilities among several (object) classes. \textit{FilterHttpContent} is the link to the restof ZAP. It extends \textit{FilterAdaptor} in the parosproxy filter package. Upon receiving an HTTP response, it only verifies the \textit{HttpMessage} has content. Then it instantiates a \textit{FilterApplyer} (concrete subclass) and calls one of its filtering methods with 2 parameters. Currently only one is available, but other methods can be added to offer various filtering algorithms. The 2 parameters are the \textit{HttpMessage} and the url of the file containing the filter terms and additional info.
\par
The \textit{FilterApplyer} then instantiates 2 helpers. One, a \textit{PageContent}, represents a given (upon creation) \textit{HttpMessage}'s content. Subclasses provide the content with the desired type, \textit{String} for the assignment. The other, a \textit{FormatFileToFilterInfo}, parses the file at given url and returns the useful information. Currently only one parsing method is available, that supports the format of the assignment and returns a \textit{Pair}. This pair consists of the weight threshold and a list of \textit{InappropriateElement} instances. \textit{InappropriateElement} models inappropriate content, of generic type, its weight and explanatory tags. To support different formats, other parsing methods can be added.
Next, the \textit{FilterApplyer} combines both result and filters the content using the found inappropriate elements.
An alternative with lower coupling would be by sending a closure from one class to the next one and never returning results.
\subsection{Iteration 4: Content Reporting}
It was created a new package to be consistent with the naming that zaproxy has. The package \textit{org.zaproxy.zap.extension.imgreport} encapsulates the classes used for the extension. The class \textit{ExtensionImageReport} extends \textit{ExtensionAdaptor} (creates, initializes and hooks a new extension), \textit{XmlReporterExtension} (gets our XML format which will be added to the zaproxy report extension) and \textit{HttpSenderListener} (converts our new extension in an observer object which is able to catch all the HttpMessages).
\par
\textit{ExtensionImageReport} validates whether \textit{HttpMessage} content is an image content, stores \textit{HttpImage} objects and delegate the creation of specific statistics format to the \textit{ImageStatistics} classes. 
\par
\textit{ImageStatisticsFactory} instantiates new concrete classes of \textit{ImageStatistics}.
\par
\textit{HttpImage} processes \textit{HttpMessage} and returns the corresponding object.
\par
\textit{ImageDimensionStatistics} is an template class implementation of \textit{ImageStatistics} used by \textit{ImageHeightStatistics}, \textit{ImageSizeStatistics} and \textit{ImageWidthStatistics}. \textit{ImageTypeStatistics} is an implementation of \textit{ImageStatistics} that creates a unique XML format.
\par
This \textit{ExtensionImageReport} was implemented as core functionality since the given XML format to the ReportExtension relies on XSL style sheets to add the new information in HTML and MarkDown reports thus those corresponding XSL files were properly updated.

\subsection{Iteration 5: Refactoring}
\subsubsection{org.parosproxy.paros.core.proxy.ExtensionLoader}
We first tackled the massive code duplication in  the \textit{ExtensionLoader} class. Most of the time, we used Java 8's support for lambda expressions and closures.
All JMenu related methods moved from \textit{ExtensionLoader} to \textit{MenuHandler}. This relieves the \textit{ExtensionLoader} from a responsibility unrelated to its main concerns, improbing cohesion.
We applied the proxy pattern and extracted the extension list and the extension map from the \textit{ExtensionLoader}. Instead, the \textit{ExtensionLoader} has an \textit{ExtensionList} field. \textit{ExtensionList} encapsulate both the list and the map. It provides most methods to manipulate them.
We also made a minor improvement to methods going over all extensions. They now use a for-each (over the private list) rather than an indexed for-loop and the public 'getExtension(i)'. The order is preserved so the behavior is the same.
patterns: factory?
\subsubsection{org.parosproxy.paros.core.proxy.ProxyThread}

\textit{Response} was created to handle the errors messages while\textit{notification} package was created to delegate all the notification method used in \textit{ProxyThread}. Due to the similarity in the algorithm, it was implemented using an abstract template class and the internal behavior was implemented in the concrete classes.
